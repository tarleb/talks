%% The following is taken from: Graph Grammer Slicing – An Approach to
%% Reducing the Combinatorial Complexity of Tracking Observables in Complex
%% Reaction System, Albert Krewinkel, Master's Thesis, 2011

\section{Integration of Quantitative Parameters}
A main goal of biological systems modeling is the gain of new insights on the
internal dynamics of the system.  The dynamics are determined by
concentrations, reaction rates and other numerical parameters which have not
been considered yet.  In order to enable quantitative analysis and simulation
of the systems, these values must be an integral part of the models.

Augmenting rule-based models with numerical parameters is straightforward, and
was already mentioned in the definition of productions for chemical systems
(see \prettyref{sec:model-react-rules} on page
\pageref{sec:model-react-rules}).  Each rewrite rule in the model, or
production in our graph grammar terminology, has a parameter assigned
determining how often or how fast the rule is executed within the system.  The
type of the parameter, reaction constant, rate or probability, may differ
depending on the intended type of simulation.

Simulation of systems becomes increasingly important the more complex the
examined system is; simulations are often possible even when the analytical
methods fail.  We describe two common techniques of reactive systems
simulation.

\subsection{Simulations Using Ordinary Differential Equations}
A common and simple simulation technique is based on mass action kinetics and
elementary reactions.  An elemental reaction occurs in a single step and
without intermediate products \cite{Svehla1993}.  The rate of an elementary
reaction is determined by mass action kinetics, that is the rate is the
product of educt concentrations and a single reaction constant.  For a
unimolecular reaction $A  \to \mathrm{products}$, the rate is given by
\[ \frac{d[A]}{dt} = - k [A] \]
and for a bimolecular reaction $A + B \to \mathrm{products}$, the rate is
\[ \frac{d[A]}{dt} = \frac{d[B]}{dt} = - k [A] [B] \]
where $k$ is the reaction constant $[A]$ and $[B]$ are the concentrations of
molecules $A$ and $B$, respectively (for details, see for example
\cite{Mortimer2003}).

Assuming all rewrite rules in a model represent elemental reactions and have
reaction constants assigned, it is possible to generate a system of ordenary
differential equations for numerical simulations.  The algorithms for this was
first described by \citet{Blinov2006} in \citeyear{Blinov2006}.  Given a set
of molecules which make up the start solution, all possible direct derivations
are determined and the corresponding differential equation is generated for
each direct derivation using mass action kinetics.  The resulting sytems can
then be analysed computationaly using standard tools of numerical integration.

This method is used by many rule-based biochemical modeling tools, including
\texttt{BioNetGen} \cite{Blinov2004}, \texttt{little b} \cite{Mallavarapu2009}
and \texttt{Bio-PEPA} \cite{Ciocchetta2008}.  The algorithm may be
computationally expensive due to the need for repeated solution of the
subgraph isomorphism problem \cite{Blinov2006} and the possibly large number
of reactions \cite{Danos2008} (see also \prettyref{sec:chall-comb-expl}).
However, complexity can be tamed using reduction techniques as described in
\prettyref{sec:meth-model-reduct}.  Once a reasonably compact ODE systems has
been generated, it allows rapid exploration of systems dynamics.

\subsection{Simulations Using Stochastic Approaches}
Stochastic simulations are an alternative to the deterministic simulations
lined out in the previous section.  The Gillespie algorithm
\cite{Gillespie1977} is a classic stochastic method to generate trajectories
of molecule concentrations within a reaction system.  Like for ODE-based
simulations described above, reactions must be elementary in order for the
original algorithm to produce correct results.  However, extension of the
algorithm have been developed such that the restriction to mass action
kinetics is lifted and more complex kinetics, like the biochemical important
Michael-Menten kinetics, may be used \cite{Rao2003,Cao2005}.

Another interesting approach are continuous time Markov chains, modeling
concentrations as discrete quantities.  This allows for probabilistic model
checking and numerical simulations which converge to the ODE case for
increasing numbers of states \cite{Calder2006}.

For both simulation types there exist modeling tools with support for the
respective method, for example \texttt{BioNetGen} \cite{Blinov2004} for the
Gillespie algorithm or \texttt{Bio-PEPA}, which supports all types of
simulations mentioned above.
